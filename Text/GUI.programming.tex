\chapter{Graphical User Interface Programming}
\vspace{-10mm}

A Graphical User Interface (GUI) of an application has to be described in its source code - mainly the layout of control elements and their interaction. Each operation system offers its own library for GUI programming. To ease the programming process, additional libraries can be used - libraries, which are focused mainly on creating application's GUI are called Widget Toolkits.

The native libraries for GUI programming are for example Windows API in Windows and xlib in Linux/Unix systems. The process of programming application's GUI with use of only native OS library is in following text explained on Windows API. Object-oriented GUI programming is explained with use of Qt\citesec{qthome} and GTK\citesec{qthome} libraries.

\section{GUI programming in Windows API}
\label{noqt}
Creating an application's GUI with use of only native OS library is quite complicated. Both Windows API and xlib are not object-oriented. The application behavior has to be solved by procedural declarations in the source code.

The GUI programming in WinAPI uses following basic elements\cite[Chapter~2]{eventloopprogramming}:

\begin{itemize}
\item Object
\item Object's Procedure
\item Message
\item Message Queue
\item Message Loop
\end{itemize}

The basic idea is following:

\begin{itemize}
\item The core of the application is the Message Loop. It loads messages from the Message Queue and redirects them to proper Objects.
\item Each Object owns a Procedure. The Procedure receives the message, process it and sends a response back to the Message Queue.
\end{itemize}

This concept is called event-driven programming. There is an example of event-driven application on Listing \ref{WinAPI}. The application loop is located on line \ref{lst:ProgramLoop}. The \clist{GetMessage} functions reads a message from the Message Queue. The \clist{DispatchMessage} forwards the message to appropriate object.

The only visible object of the example application is a window created on line \ref{lst:window}. The procedure of the window, which solves its received messages is defined on line \ref{lst:WndProc}.

The \clist{main} function of the application is replaced by the \clist{WinMain} function. The application course is following: the window is created and the loop starts running; if the user clicks on the window, a message is added to the queue; the message is read from the queue and then forwarded to the application itself or to some object.\cite[Chapter~2]{eventloopprogramming}

\begin{lstlisting}[label=WinAPI,caption={An example of a simple application using Windows API for GUI rendering.\protect\cite{WinAPIexample}},escapeinside={@}{@}]
@\label{lst:WndProc}@LRESULT CALLBACK WndProc(HWND hwnd, UINT msg, WPARAM wParam, LPARAM lParam){
    switch(msg){
        case WM_CLOSE:
            DestroyWindow(hwnd);
        break;
        case WM_DESTROY:
            PostQuitMessage(0);
        break;
        default:
            return DefWindowProc(hwnd, msg, wParam, lParam);
    }
    return 0;
}
@\label{lst:WinMain}@int WINAPI WinMain(HINSTANCE hInstance, HINSTANCE hPrevInstance,
    LPSTR lpCmdLine, int nCmdShow){
    WNDCLASSEX wc;
    HWND hwnd;
    MSG Msg;
    //Step 1: Registering the Window Class
    wc.cbSize        = sizeof(WNDCLASSEX);
    wc.style         = 0;
    wc.lpfnWndProc   = WndProc;
    wc.cbClsExtra    = 0;
    wc.cbWndExtra    = 0;
    wc.hInstance     = hInstance;
    wc.hIcon         = LoadIcon(NULL, IDI_APPLICATION);
    wc.hCursor       = LoadCursor(NULL, IDC_ARROW);
    wc.hbrBackground = (HBRUSH)(COLOR_WINDOW+1);
    wc.lpszMenuName  = NULL;
    wc.lpszClassName = g_szClassName;
    wc.hIconSm       = LoadIcon(NULL, IDI_APPLICATION);
    // Step 2: Creating the Window
 @\label{lst:window}@   hwnd = CreateWindowEx(
        WS_EX_CLIENTEDGE,
        g_szClassName,
        "The title of my window",
        WS_OVERLAPPEDWINDOW,
        CW_USEDEFAULT, CW_USEDEFAULT, 240, 120,
        NULL, NULL, hInstance, NULL);
    // Step 3: The Message Loop
@\label{lst:ProgramLoop}@    while(GetMessage(&Msg, NULL, 0, 0) > 0){
        TranslateMessage(&Msg);
        DispatchMessage(&Msg);
    }
    return Msg.wParam;
}
\end{lstlisting}

Programming GUI applications in plain WinAPI requires necessary steps to be done, such as: defining the main loop, assigning a procedure to the object, etc. All these steps are ensured by Widget Toolkits, if used. 

\begin{comment}

\section{GUI programming in Windows API}
Development of GUI application for Windows is possible with Windows API. This is the name given to a set of various APIs - communication interfaces for performing services. These communication interfaces offer manipulation with file-system, devices and also creating graphic windows.

Most common parts of Windows API are:
\begin{itemize}
\item \clist{kernel32.dll}, which is needed for memory management, input/output, process and thread creation.
\item \clist{user32.dll} for creating GUI elements, such as windows, buttons, etc.
\item \clist{shell32.dll} which offers access to Windows command line
\item \clist{WinSock} to use network
\end{itemize}
\end{comment}

\section{Widgets Toolkits}
\label{guiframeworks}
Widgets Toolkits are libraries designed to ease GUI application creation. Complicated widget declaration processes are encapsulated inside the library. Then, library classes are used instead of calling OS related functions. Examples of currently popular Widgets Toolkits can be Qt\citesec{qthome} and GTK\citesec{gtkhome} libraries\cite[pages~413-416]{qtvsgtk}.

Both Qt and GTK are based on the same schema. All control elements are represented by library classes. The control elements can be easily added to application windows. Functions describing application response to user actions are defined. Then the control elements and the functions are connected together.

The structure of an application using Qt or GTK is very similar. Examples of two identical applications using GTK and Qt can be found on Listings \ref{gtk} and \ref{qt}. Both applications create a simple window with a button inside. Clicking the button writes a console output. The button and window creation can be found on lines \ref{lst:gtkbuttonstart} - \ref{lst:gtkbuttonfinish} on Listing \ref{gtk} (GTK) and lines \ref{lst:qtbuttonstart} - \ref{lst:qtbuttonfinish} on Listing \ref{qt} (Qt). The process is very similar unlike Qt uses functions in its classes and GTK uses global functions. The command on lines \ref{lst:gtkconnect} (Listing \ref{gtk}) and \ref{lst:qtconnect} (Listing \ref{qt}) is used for setting up interaction between button click and proper function call. The main difference is that GTK allows connection to global function calls (line \ref{lst:gtkcall}, Listing \ref{gtk}), unlike Qt allows calling functions of object inheriting QObject class (line \ref{lst:gtkcall}, Listing \ref{qt}).

\begin{lstlisting}[label=gtk,caption={A sample application using GTK library.\protect\cite{gtktutorial}},escapeinside={@}{@}]
@\label{lst:gtkcall}@static void message(GtkWidget *widget, gpointer data){
    cout << "Hello World!";
}
...
@\label{lst:gtkbuttonstart}@GtkWidget *window = gtk_window_new (GTK_WINDOW_TOPLEVEL);
GtkWidget *button = gtk_button_new_with_label ("Hello World");
GtkWidget *layout = gtk_fixed_new();
gtk_fixed_put(GTK_FIXED(layout), button);
@\label{lst:gtkbuttonfinish}@gtk_container_add(GTK_CONTAINER(window), layout);
@\label{lst:gtkconnect}@g_signal_connect (button, "clicked", G_CALLBACK (message), NULL);
\end{lstlisting}

\begin{lstlisting}[label=qt,caption={A sample application using Qt library.},escapeinside={@}{@}]
@\label{lst:qtcall}@class callhandler : public QObject {
Q_OBJECT
public slots:
	void message () {
		cout << "Hello World!";
	}
}
...
@\label{lst:qtbuttonstart}@QWidget *window = new QWidget();
QPushButton *button = new QPushButton();
QLayout *layout = new QGridLayout();
layout->add(button);
@\label{lst:qtbuttonfinish}@window->setLayout(layout);
@\label{lst:qtconnect}@connect(button,SIGNAL(clicked()),callhandler,SLOT(message()));
\end{lstlisting}
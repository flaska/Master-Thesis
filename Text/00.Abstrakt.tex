\thispagestyle{empty}

% příprava:\usepackage{subfig}
\newbox\odstavecbox
\newlength\vyskaodstavce
\newcommand\odstavec[2]{
    \setbox\odstavecbox=\hbox{
         \parbox[t]{#1}{#2\vrule width 0pt depth 4pt}}
    \global\vyskaodstavce=\dp\odstavecbox
    \box\odstavecbox}
\newcommand{\delka}{120mm}


\newcommand{\pracovisteVed}{\km,\\ \fjfi,\\ \cvut}

\newcommand{\konzultant}{}
\newcommand{\pracovisteKonz}{}

\newcommand{\klicova}{grafické uživatelské rozhraní, lékařské vizualizace, objektově orientované programování, C++, Qt, DICOM}
\newcommand{\keywords}{graphical user interface, medical imaging, object-oriented programming, C++, Qt, DICOM}   



{\noindent \bf \large Abstract} \\[5mm]
\begin{tabular}{l p{10cm}}
	{\em Master's Thesis}	& 	\\[1mm]
	{\em Title:}	& \nazeven	\\[1mm]
	{\em Author:}	& \autor	\\[1mm]
	{\em Program:} 	& \obor		\\[1mm]
	{\em Supervisor:}& \vedouci	\\
				& \km		\\
				& \fjfi		\\
				& \cvut		\\[1mm]
	{\em Keywords:}	& \odstavec{\delka}{\keywords}	\\
\end{tabular}

This Diploma Thesis describes the development of a C++ application, which is used for displaying images captured on a Magnetic Resonance Imaging unit. Prior to this work, the application was partly implemented. This work sets out a few goals: redesign the rendering part of the application, implement Multi-planar recontruction and a system for additional plugins. In addition, the thesis discusses GUI applications programming and compilation of C++ applications in Win32.


\vspace{10mm}
{\noindent \bf \large Abstrakt} \\[5mm]
\begin{tabular}{l p{10cm}}
	{\em Diplomová práce}	& 	\\[1mm]
	{\em Název:}	& \nazevcz	\\[1mm]
	{\em Autor:}	& \autor	\\[1mm]
	{\em Keywords:}	& \odstavec{\delka}{\klicova}	\\
\end{tabular}

Tato diplomová práce se zabývá vývojem programu sloužícího pro zobrazování snímků z magnetické resonance. Na začátku této práce již byla aplikace částečně implementována. Tato diplomová práce se snaží zejména o následující úkoly: kompletně přepsat část aplikace věnující se vykreslování; implementovan systém zobrazení označovaný jako multiplanární rekonstrukce; přidat do aplikace rozhraní umožňující používání přídavných modulů k aplikaci. Dále se pak práce věnuje v obecnější rovině programování aplikací s grafickým uživatelským rozhraním a shrnuje poznatky o překladu C++ aplikací v prostředí Win32.






\chapter{Introduction}
\vspace{-10mm}
The aim of this Diploma Thesis is development of a C++ application used for viewing Magnetic Resonance data. The task was given by IKEM institute\footnote{Institute for Clinical and Experimental Medicine} and was started in work \cite{neskudla} and followed in works \cite{flaska_bc} and \cite{flaska_vu}. The application (further called ``Dicom-Presenter'') allows to open MRI images and offers unique displaying features required by IKEM.

This Diploma Thesis sets three goals to be done: handle project compilation, rewrite application rendering engine and implement new features. The project compilation process needed to be reviewed and automated. There were application dependencies on five external libraries which complicated project deployment. Therefore, the rendering part of the application had to be rewritten to remove project dependency on some external libraries (OpenGL, Cg toolkit, plib). Lastly, multi-planar reconstruction\footnote{A multi-planar reconstruction is a way for displaying three-dimensional images. Three slices in three perpendicular planes are displayed. See Chapter \ref{multiplanar}.} and image segmentation were added into the application.

\chapter{Debugging Applications Source Code}
\vspace{-10mm}
It is natural, that an appplication like the Dicom-Presenter cannot work seamlessly from the first moment. The application was developed on a specific hardware configuration and tested on a limited set of input data. The next step in software development after implementation is usually testing, which helps to detect problems, which could not occur on development machines and input data.

A few of various problems appeared in the first implementation of the Dicom-Presenter, done by my predecessor\cite{neskudla}, these had to be removed. This chapter focuses on several problems, which were treated.

\section{Unsuccesful Object Instantiation}
Dicom-Presenter suffered from some issues, which appeared as ``Segmentation Fault'' on Linux OS and ``Access Violation'' on Windows OS. Both names describe an issue, when the application tries to access a part of the memory, which is not reserved for it. The problem occurs very often in C++ environment. Mostly it is caused by an unsuccesfull object instantiation. A few objects in Dicom-Presenter, representing GUI icons, were instanced with a constructor, which took a filepath as an argument:

\clist{iCloseIcon = QGLWidget::bindTexture(QImage("closeicon.bmp"));}

If the file referenced was inaccessible by the application, the instantiation process failed. The above example results in the fact that the \clist{iCloseIcon} pointer remains unchanged by the \clist{QGLWidget::bindTexture} function. Therefore, a NULL value remains in the pointer. If the application tries to access the uncreated object, it fails instantly.

The solution is to test the pointer before each attempt to access the referenced object. The necessary condition is, that the pointer  must be initialized to NULL value right after its declaration. If the pointer is not initialized this way, there is no simple possibility to test its incorrectness. If a pointer is not initialized, it contains some random value, which is an address pointing to some space in the computer memory.

\section{Debugging Graphic Output}
The Dicom-Presenter suffered from a few issues in its rendering part. The graphic output instantly disappeared, when pixel-shaders\footnote{Pixel-shader is a program compiled and runned at GPU. It can be used to effectively produce special visual efects such as shading, bump-mapping, etc.} were used (on computers using nVidia GPUs). Nevertheless all initializations related to Cg Toolkit library were completed correctly. The Cg Toolkit library can run in various configurations - from hardware specific to general, usable on most GPUs. None of the tested configurations solved the problem.

The visual output of Dicom-Presenter is combined from smaller units, which are added during several steps. The displayed images must be firstly obtained from a three-dimensional texture, color adjusted, then rendered into a Workspace's framebuffer, then the Workspace framebuffer was rendered into application's framebuffer, other objects were added and finally the application's framebuffer was rendered onto the computer screen. The image, which disappeared could be lost during any of the phases.

When an ordinary application is debugged, the values of all variables can be tracked. A graphic output of an application, which is being prepared in the computer memory, cannot be tracked that easily. To watch the graphic output continuously, a special test class was created. It attached the graphic output of the application to its own framebuffer. The framebuffer was exported to a file. So, the graphic output could be observed in any place of the application. On the other hand, since the output was attached to the framebuffer, it was not rendered onto the computer screen.

\section{Real Data Manipulation}
When an application is developed, it is usually tested on some limited set of input data. The Dicom-Presenter was tested only on MRI studies captured in IKEM. The problem was, that all the images were captured on the same MRI device. Therefore, the file header had the same format in all pictures. When the Dicom-Presenter was tested on more diverse set of data, gained from various web-pages oriented on medical imaging, the application failed to read half of the data.

After stack-tracing the application, the exact place of the problem became apparent. The application failed while calling a constructor from DCMTK library:

\clist{iDicomImage = new DicomImage(fileName)}

The problem appeared somewhere inside the external library. The only parameter given to the library is a string with file-path, which cannot be misleading. To recognize the problem, the external library was compiled in Debug mode and stack-traced. Afterwards, the problem could be identified - the DCMTK library could not recognize file headers of the images.

GDCM\citesec{gdcm_home} is a an application, which could solve the problem - it allows conversion of DICOM files' header. The conversion performed on an un-readable image, repaired the file header that way, that the image could be fully usable in Dicom-Presenter. The question was how to import the external tool into Dicom-Presenter. Three options were possible:

\begin{itemize}
\item Compile GDCM as a static or dynamic library and link against the library, to use its function inside Dicom-Presenter.
\item Compile GDCM as a binary application and use GDCM in Dicom-Presenter as an external tool.
\item Compile GDCM as shared library and load the library at a run-time to use its functions.
\end{itemize}

The first option was the fastest - using the library's functions right inside the application had the least effect on the application's performance. The problem in this case is, that GDCM library must be present at compilation time or at run-time (static or dynamic library), even if it is unused. It is a limiting condition, which could affect the application's portability. The second and the third option are very simillar. GDCM library is treated as an external tool - it's absence will not prevent the user from using the application (outside GDCM related tasks). The GDCM library was finally added as an external binary, since the manipulation is easier. A \clist{system()} function from Windows  API allows running any command from OS command line, so the GDCM binary could be called.

\begin{lstlisting}[caption={A conversion of DICOM files header using GDCM as external tool.}]
bool CDicomFrames::LoadDicomImage(char *fileName, bool isFirst, int framesCount) {
	iDicomImage = new DicomImage (fileName);
	EI_Status imageStatus = iDicomImage->getStatus();

	if (imageStatus==EIS_MissingAttribute){			
		QString strCommand("gdcmconv --raw --force ");
		strCommand.append(fileName);
		strCommand.append(" temp.dcm");
		int i = system(strCommand.toAscii().data());
		if(i==0){
			iDicomImage=NULL;
			iDicomImage = new DicomImage ("temp.dcm");
			remove("temp.dcm");
		}else{
			QMessageBox msgBox;
			msgBox.setText("DICOM file is missing some header attribute.");
			msgBox.exec();
		}		
	}
 ...
}
\end{lstlisting}
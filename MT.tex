%%%%%%%%%%%%%%%%%%%%%%%%%%%%%%%%%%%%%%%%%
% BAKALÁRSKÁ PRÁCE			%
% JAKUB FLAŠKA				%
% Šablona prevzata ze stránek KSE	%
% (C) FJFI CVUT v Praze			%
%%%%%%%%%%%%%%%%%%%%%%%%%%%%%%%%%%%%%%%%%

% Typ dokumentu
\documentclass[a4paper,12pt]{report}	% report - jednostranný tisk

%%%%%%%%%%%%%%%%%%%%%%%%%%%%%%%%%%%%%%%%%%%%%%%%%%%
%% Pouzite balícky

% Kódování a zpracování CJ
%\usepackage[czech]{babel}	% cestina
\usepackage[T1]{fontenc}	% balicek fontu
\usepackage[utf8]{inputenc}	% cestina
% Vytvorení indexu a seznamu použité literatury
\usepackage{index}		% vytvorení obsahu
\usepackage[pdftex]{hyperref}	% vygeneruje rejstrík pri použití pdflatex
\usepackage{cite}		% vytvorení literatury
\usepackage{multibib}		% více zdroju literatury
% Práce s obrázky
\usepackage{graphicx}		% obrázky
\usepackage{subfig}		% více obrázku v polícku
% Ostatní
\usepackage{listings}		% vkladani zdrojoveho kodu
\usepackage[usenames]{color}	% použití barevného textu
\usepackage{xcolor}
\usepackage{url}		% zpracování www adresy
\usepackage{verbatim}		% moznost viceradkovych kommentaru prikazem \begin{comment}
\usepackage{array}		%
\usepackage{caption}		% Popisky obrázku jiným písmem, než zbytek textu.
\usepackage{amsmath}
\usepackage[all]{hypcap}
\usepackage{caption}
\usepackage{pdfpages} 


%%%%%%%%%%%%%%%%%%%%%%%%%%%%%%%%%%%%%%%%%%%%%%%%%%%
%% Formát textu

\oddsidemargin=10mm		% levý okraj
\topmargin=-15mm		% horní okraj
\textwidth=150mm
\textheight=240mm
\pagenumbering{arabic}
\pagestyle{plain}

%\parindent=0pt			% odsazení prvního rádku
\parskip=7pt			% mezera mezi odstavci
\frenchspacing			% typografická pravidla

\renewcommand{\rmdefault}{phv}	% Arial
\renewcommand{\sfdefault}{phv}	% Arial


%%%%%%%%%%%%%%%%%%%%%%%%%%%%%%%%%%%%%%%%%%%%%%%%%%%
% Nastavení balícku
%%%%%%%%%%%%%%%%%%%%%%%%%%%%%%%%%%%%%%%%%%%%%%%%%%%


%%%%%%%%%%%%%%%%%%%%%%%%%%%%%%%%%%%%%%%%%%%%%%%%%%%%
%% Formátováni zdroju - cite, Multibib
%Bibtex
\bibliographystyle{plain}		% styl primarnich zdroju

\newcites{sec}{Secondary sources}	% sekundarni zdroje
\bibliographystylesec{plain}		% styl sekundarnich zdroju

%%%%%%%%%%%%%%%%%%%%%%%%%%%%%%%%%%%%%%%%%%%%%%%%%%%
% Vytvorení indexu pro rejstrík a citace - index
%index
\newindex{default}{idx}{ind}{}

%%%%%%%%%%%%%%%%%%%%%%%%%%%%%%%%%%%%%%%%%%%%%%%%%%%
%%%%%%%%%%%%%%%%%%%%%%%%%%%%%%%%%%%%%%%%%%%%%%%%%%%

\DeclareFontShape{OT1}{cmtt}{bx}{n}{cmttb10}{}	% Definování fontu pro bold type-writer.

% Caption
\captionsetup{%font=small,		% Formát popisku.
	format=plain,
	labelfont=bf,
	%textfont=it
}



% Nastavení odkazu - hyperref
\hypersetup{ 
linkbordercolor={1 1 1},	% rámecek kolem odkazu bude bílý
citebordercolor={1 1 1}		% rámecek kolem odkazu citace bude bílý 
} 

% Nastavení balícku pro vkládání zdrojového kódu - lstlistings
\definecolor{LightGray}{RGB}{245,245,245}
%\definecolor{LightRed}{RGB}{255,100,100}
%\definecolor{LightGreen}{RGB}{70,150,60}
%\definecolor{LightBlue}{RGB}{80,100,240}

\definecolor{LightRed}{RGB}{255,100,100}
\definecolor{LightGreen}{RGB}{60,143,49}
\definecolor{LightBlue}{RGB}{39,62,237}
\definecolor{Purple}{RGB}{162,4,207}

\newcommand{\red}[1]{\textcolor{red}{#1}}
\newcommand{\orange}[1]{\textcolor{orange}{#1}}
\newcommand{\lightgreen}[1]{\textcolor{LightGreen}{#1}}
\newcommand{\lightblue}[1]{\textcolor{LightBlue}{#1}}

\lstset{ %
language=C++,                % choose the language of the code
basicstyle=\small\tt\color{black},          % print whole listing small
keywordstyle=\small\color{LightBlue},	% bold black keywords
identifierstyle=\small\color{black},           % nothing happens
commentstyle=\small\color{black}, % white comments
stringstyle=\ttfamily,      % typewriter type for strings
showstringspaces=false,     % no special string spaces
numbers=left,                   % where to put the line-numbers
numberstyle=\tiny\tt,      % the size of the fonts that are used for the line-numbers
%stepnumber=2,                   % the step between two line-numbers. If it's 1 each line will be numbered
numbersep=5pt,                  % how far the line-numbers are from the code
%backgroundcolor=\color{LightGray},  % choose the background color. You must add \usepackage{color}
showspaces=false,               % show spaces adding particular underscores
showstringspaces=false,         % underline spaces within strings
showtabs=false,                 % show tabs within strings adding particular underscores
frame=single,			% adds a frame around the code
tabsize=3,	                % sets default tabsize to 2 spaces
%captionpos=b,                   % sets the caption-position to bottom
breaklines=true,                % sets automatic line breaking
breakatwhitespace=false,        % sets if automatic breaks should only happen at whitespace
%title={Zdrojovy kod},                 % show the filename of files included with \lstinputlisting; also try caption instead of title
%escapeinside={\%*}{*)}          % if you want to add a comment within your code
escapechar=!,
}

%\renewcommand{\lstlistingname}{Kód}


\renewcommand{\textfraction}{0.05}
\renewcommand{\topfraction}{0.2}	% max fraction of floats at top
\renewcommand{\bottomfraction}{0.2}	% max fraction of floats at bottom

%\setlength{\topsep}{10pt}
%\setlength{\itemsep}{10pt}

\newcommand{\redlist}[1]{{\color{LightRed}#1}}
\newcommand{\greenlist}[1]{{\color{LightGreen}#1}}

%%%%%%%%%%%%%%%%%%%%%%%%%%%%%%%%%%%%%%%%%%%%%%%%

% Formátování C++ príkazu uprostred textu
\newcommand{\clist}[1]{\texttt{\hyphenchar\font45\relax #1}} % font s fixní vzdáleností

% Makro pro ceské uvozovky, použití \uv{...}
\def\bq{\mbox{\kern.1ex\protect\raisebox{-1.3ex}[0pt][0pt]{''}\kern-.1ex}}
\def\eq{\mbox{\kern-.1ex``\kern.1ex}}
\def\ifundefined#1{\expandafter\ifx\csname#1\endcsname\relax }%
\ifundefined{uv}%
        \gdef\uv#1{\bq #1\eq}
\fi

\hyphenation{Open-GL}

%%%%%%%%%%%%%%%%%%%%%%%%%%%%%%%%%%%%%%%%%%%%%%%%%%%%%%
%%%%%%%%%%%%%%%%%%%%%% BAKALARSKA PRACE %%%%%%%%%%%%%%
%%%%%%%%%%%%%%%%%%%%%%%%%%%%%%%%%%%%%%%%%%%%%%%%%%%%%%

%%%%%%%%%%%%%%%%%%%%%%%%%%%%%%%%%%%%%%%%%%%%%%%%%%%%
%%%%%%%%%%%%%%%%%%%%%% POJMY  %%%%%%%%%%%%%%%%%%%%%%            

\newcommand{\cvut}{Czech Technical University in Prague}
\newcommand{\fjfi}{Faculty of Nuclear Sciences and Physical Engineering}
\newcommand{\km}{Department of Mathematics}
\newcommand{\obor}{Inženýrská informatika}
\newcommand{\zamereni}{Tvorba software}
\newcommand{\nazevcz}{Softwarový nástroj pro manipulaci s daty z magnetické rezonance a jejich vizalizaci}
\newcommand{\nazeven}{Development of a Software Instrument for MRI Data Manipulation and Visualization }     
\newcommand{\autor}{Bc.~Jakub Flaška}
\newcommand{\rok}{2012}
\newcommand{\vedouci}{Ing.~Pavel Strachota} 

%%%%%%%%%%%%%%%%%%%%%%%%%%%%%%%%%%%%%%%%%%%%%%%%

%%%%%%%%%%%%%%%%%%%%%% UVODNI STRANA  %%%%%%%%%%%%%%%%%%%%%%

\begin{document}
%\begin{comment}

\thispagestyle{empty}

\begin{center}
	{\fontsize{15}{15.5} \bf \cvut\\[2mm] \fjfi \\[2mm] \km}	% Název školy, fakulta
	\vfill
		\begin{center}
			\includegraphics[width=50mm]{Text/IMG/00_Logo_CVUT_bw.jpg}	% Logo
		\end{center}
	\vfill
		{\fontsize{35}{36.5} \bf MASTER'S THESIS}		% BAKALÁRSKÁ PRÁCE
	\vfill			
		{\fontsize{20}{20.5} \bf \nazeven}	% Název práce
	\vfill	
		{\large %\em %\bf								
			\begin{tabular}{rl}
				Author 		& {\bf	\autor}			\\					% Autor
				Supervisor 	& {\bf	\vedouci }		\\					% Školitel
				Year		& {\bf	\rok }			\\					% Rok
			\end{tabular}
		}
\end{center}


%%%%%%%%%%%%%%%%%%%%%% OBSAH %%%%%%%%%%%%%%%%%%%%%%
\newpage
\tableofcontents


%\begin{comment}
%%%%%%%%%%%%%%%%%%%%%% PROSTOR PRO ZADANI  %%%%%%%%%%%%%%%%%%%%%%
\newpage
\thispagestyle{empty} 

\includepdf[pages=-]{Text/IMG/zadani.pdf} 

%%%%%%%%%%%%%%%%%%%%%% PROHLASENI %%%%%%%%%%%%%%%%%%%%%%
\newpage
\thispagestyle{empty}

~
\vfill % prázdné místo

{\bf Author's Statement}

\vspace{0.5cm} % vertikální mezera

Hereby, I certify that this diploma thesis is my own work. I have not assumed any of the texts, all the text are written by my person. 

I am not against using this diploma thesis by Czech Technical University in any way described in ``Zakon autorsky'' about school projects.

Noone except the author and Czech Technical University has a right to manipulate with the text - noone can copy the text or part of the text and use it in another work.

I am not against redistributing the whole text - if the text will be distributet together with the initial page and its autorship will be clear.

\vspace{10mm}

Tímto prohlašuji, že tato diplomová práce je mým autorským dílem. Texty v práci nejsou prevzaté, ale jsou psané autorem.

Nemám závažný duvod proti použití tohoto díla Ceským Vysokým Ucením Technickým tak, jak to vyplívá z autorského zákona v textu o školním díle. 

Kdokoliv mimo autora a Ceské Vysoké Ucení Technické pak nemá právo s textem manipulovat - zejména pak celý text, nebo cást textu vydávat za autorské dílo nekoho jiného.

Umožnuji pak šírení kompletního textu této diplomové práce, pokud bude text šíren jako celek vcetne úvodní strany a nebude pochybnost o jeho autorství. 

\vspace{10mm}Praha, June 5, 2012\hfill
	\begin{tabular}{c}
	\includegraphics[width=50mm]{Text/IMG/podpis.jpg}\\ 
	\autor
	\end{tabular}


%%%%%%%%%%%%%%%%%%%%%% PODEKOVANI %%%%%%%%%%%%%%%%%%%%%%
%\end{comment}
%\begin{comment}%%%%%%%%%%%%
\newpage
\thispagestyle{empty}

~
\vfill % prázdné místo

{\bf Acknowledgment / Podekování}

\vspace{5mm} % vertikální mezera

I would really like to thank my supervisor Ing. Pavel Strachota for his outstanding help while creating this diploma thesis. His suggestions greatly improved the text.

Rád bych podekoval svému školiteli Ing. Pavlu Strachotovi za vedení této diplomové práce. Jeho pripomínky byly znacne podnetné a výrazne prispely ke zkvalitnení textu.

\begin{flushright}
Author
\end{flushright}


%%%%%%%%%%%%%%%%%%%%%% ABSTRAKT %%%%%%%%%%%%%%%%%
%\end{comment}%%%%%%%%%%%
\begin{comment}
\newpage
\thispagestyle{empty}

% příprava:\usepackage{subfig}
\newbox\odstavecbox
\newlength\vyskaodstavce
\newcommand\odstavec[2]{
    \setbox\odstavecbox=\hbox{
         \parbox[t]{#1}{#2\vrule width 0pt depth 4pt}}
    \global\vyskaodstavce=\dp\odstavecbox
    \box\odstavecbox}
\newcommand{\delka}{120mm}


\newcommand{\pracovisteVed}{\km,\\ \fjfi,\\ \cvut}

\newcommand{\konzultant}{}
\newcommand{\pracovisteKonz}{}

\newcommand{\klicova}{grafické uživatelské rozhraní, lékařské vizualizace, objektově orientované programování, C++, Qt, DICOM}
\newcommand{\keywords}{graphical user interface, medical imaging, object-oriented programming, C++, Qt, DICOM}   



{\noindent \bf \large Abstract} \\[5mm]
\begin{tabular}{l p{10cm}}
	{\em Master's Thesis}	& 	\\[1mm]
	{\em Title:}	& \nazeven	\\[1mm]
	{\em Author:}	& \autor	\\[1mm]
	{\em Program:} 	& \obor		\\[1mm]
	{\em Supervisor:}& \vedouci	\\
				& \km		\\
				& \fjfi		\\
				& \cvut		\\[1mm]
	{\em Keywords:}	& \odstavec{\delka}{\keywords}	\\
\end{tabular}

This Diploma Thesis describes the development of a C++ application, which is used for displaying images captured on a Magnetic Resonance Imaging unit. Prior to this work, the application was partly implemented. This work sets out a few goals: redesign the rendering part of the application, implement Multi-planar recontruction and a system for additional plugins. In addition, the thesis discusses GUI applications programming and compilation of C++ applications in Win32.


\vspace{10mm}
{\noindent \bf \large Abstrakt} \\[5mm]
\begin{tabular}{l p{10cm}}
	{\em Diplomová práce}	& 	\\[1mm]
	{\em Název:}	& \nazevcz	\\[1mm]
	{\em Autor:}	& \autor	\\[1mm]
	{\em Keywords:}	& \odstavec{\delka}{\klicova}	\\
\end{tabular}

Tato diplomová práce se zabývá vývojem programu sloužícího pro zobrazování snímků z magnetické resonance. Na začátku této práce již byla aplikace částečně implementována. Tato diplomová práce se snaží zejména o následující úkoly: kompletně přepsat část aplikace věnující se vykreslování; implementovan systém zobrazení označovaný jako multiplanární rekonstrukce; přidat do aplikace rozhraní umožňující používání přídavných modulů k aplikaci. Dále se pak práce věnuje v obecnější rovině programování aplikací s grafickým uživatelským rozhraním a shrnuje poznatky o překladu C++ aplikací v prostředí Win32.






\end{comment}


%%%%%%%%%%%%%%%%%%%%%%  TEXT PRÁCE %%%%%%%%%%%%%%%%%%%%%%%%%%%%%%%%%%%%%%%%%%%%

\chapter{Introduction}
\vspace{-10mm}




\chapter{Dicom-Presenter}
\vspace{-10mm}
My predecessor Bc. Pavel Neskudla started development of an application for viewing images captured on Magnetic Resonance Imaging (MRI) unit in his Master's thesis\cite{neskudla}. It was a task given by IKEM institute in Prague. IKEM institute specialists would utilize some application deployable on personal computers which would allow displaying MRI images. It is possible to find such applications, but unfortunately those are too expensive commercial solutions or otherwise freeware applications and they do not reach quality requirements\footnote{See \cite[page~9]{flaska_bc} for an overview of accessible DICOM image viewers}. Therefore, IKEM asked for development of such an application, which would match their needs. Moreover, in this case, the sponsor was allowed to raise unique functionality demands not accessible in other programs (See section \ref{requirements}).

\section{DICOM standard}

DICOM is an abbreviation of Digital Imaging and Communications in Medicine, it is a standard for handling, storing and transmitting medical information. \footnote{DICOM standard was created by  U.S. National Electrical Manufacturers Association\citesec{nema}.} Namely, DICOM standard describes a file format for storing medical data and besides it describes a protocol for exchanging this data. DICOM standard uses common IT standards such as JPEG, TCP/IP, etc.

The main reason why DICOM was created is to avoid medical data confusion. DICOM files are equipped with a header including patient's information. This permanent attachment of header data to DICOM files should avoid random data substitution. There is information about a patient, about a medical facility and about a diagnosis in a file header.

\subsection{DICOM viewers}
\label{viewers}
DICOM software is very consumer-specific. Therefore, there are just a few couples of DICOM image viewers available. According to \citesec{idoimaging} these are the most used DICOM viewers:

\begin{itemize}
  \setlength{\itemsep}{0pt}
  \setlength{\parskip}{0pt}
  \setlength{\parsep}{0pt}
\item \emph{Myrian}, Intrasense, \url{http://www.intrasense.fr/}
\item \emph{NovaPACS}, Novarad, \url{http://www.novapacs.com/}
\item \emph{K-Pacs}, Dr. med. Andreas Knopke, \url{http://www.k-pacs.net/}
\item \emph{DICOM Works}, Philippe PUECH, Loïc BOUSSEL, \url{http://dicom.online.fr/}
\item \emph{OsiriX}, OsiriX Foundation, \url{http://www.osirix-viewer.com/}
\item \emph{Aeskulap}, Alexander Pipelka, \url{http://aeskulap.nongnu.org/}
\item \emph{kradview}, David Santo Orcero, \url{http://www.orcero.org/irbis/kradview/}
\item \emph{SureVistaVision\texttrademark DICOM Viewer}, MS Technology, \url{http://www.ms-technology.com/medical-solutions/sure-vista-vision.html}
\item \emph{UniPACS},  \url{http://www.idoimaging.com/}
\item \emph{syngo Imaging}, Siemens, \url{http://www.medical.siemens.com/}
\item \emph{VR-Render}, IRCAD, \url{http://www.ircad.fr/softwares/vr-render/Software.php}
\item \emph{MicroDicom}, Simeon Antonov Stoykov, \url{http://www.microdicom.com/}
\end{itemize}

Unfortunately, Myrian, NovaPACS, syngo Imaging and SureVistaVision\texttrademark are commercial applications. Therefore, they are not suitable for IKEM use. OsiriX is very powerful DICOM viewer but it is available only for Max OS X. Thus the variety of DICOM viewers is very limited.

All mentioned DICOM viewers do have very similar appearance. A program GUI is divided into a viewing area and a control area. The application can display a DICOM image in the viewing area or it can display a multi-planar reconstruction of the image. The control area offers image enhancing operations with the image displayed in the viewing area.

\begin{table}[ht]
	\caption{DICOM viewers.}
	\centering
	\begin{tabular}{cc}
			\includegraphics[width=0.5\textwidth,height=0.375\textwidth]{Text/IMG/01_Siemens.jpg}
		&
			\includegraphics[width=0.5\textwidth,height=0.375\textwidth]{Text/IMG/01_Myrian.jpg}
		\\
			syngo Imaging~\citesec{siemens} & Myrian~\citesec{intrasense}	
		\\
			\includegraphics[width=0.5\textwidth,height=0.375\textwidth]{Text/IMG/01_OsiriX.jpg}
		&
			\includegraphics[width=0.5\textwidth,height=0.375\textwidth]{Text/IMG/01_UniPACS.jpg}
		\\
			OsiriX~\citesec{osirix} & UniPACS~\citesec{unipacs}
		\\
		\end{tabular}
\end{table}%




\section{Application Requirements}
\label{requirements}
The IKEM specialists asked for a typical DICOM images viewer with few more specific features which they missed in freeware programs. A typical DICOM viewer allows opening .dcm files and displaing them. .dcm file in this case is a jpeg image equipped with special header including patient's information. MRI images are three-dimensional, thus user has to be able to select, which slice to be displayed. Some DICOM viewers offer a multi-planar image reconstruction. Other offered functions may vary.

There have been two specific requirements on application functionality by IKEM specialists. These were not available in freeware DICOM applications. The most important function was the possibility to open several images at once and display them on one screen. The user should be allowed to arrange images on screen to any possible layout he/she prefers. This functionality allows physicians to see two or more different MRI images on screen so they can easily determine pathological differences among observed organs. It is useful for studying, or teaching.

There have been also requirements that the application should be able to record user's manipulation with images as a video. Then physician can prepare his presentation of images at home and then play the video to colleagues.

\section{Implementation}
After summarizing all the application requirements given by IKEM it was needed to choose proper technologies for application implementation. The aim was to develop a GUI application capable to open and display images, there are several ways to go, so it was the place for a discussion. 

The main question is which programming language to choose for the implementation. C\#, Java, C++ are the most popular programming languages for extensive GUI applications. C++ offers low level features and thus it offers very good performance. Java and C\# are both interpreted languages. Both were designed to be simple and robust. Although C\# and Java provide performance similar to C++ in overall tasks, still C++ reaches better performance in tasks related to imaging. Therefore, C++ was chosen as a programming language of Dicom-Presenter.

A Graphic User Interface of a C++ application is generally written with use of some library. It is possible to write a C++ GUI interface without external library using only OS related function calls but unfortunately this solution brings too complicated source code\footnote{A comparison of GUI interface written with and without external library can be found in Section \ref{noqt}}. There are more than ten application frameworks possible to be used for writing C++ GUI application. Qt and GTK are the most popular libraries used in many commercial applications and offering extensive possibilities for C++ application programming\cite{xxx}. Besides GUI creation Qt and GTK offers assistance with image manipulation, network access, xml parsing, etc. Both libraries are cross-platform. Finally, Qt was used to better documentation offered by its author.

There was a research idea when creating Dicom-Presenter: check possibility of using GPU in this kind of application. Dicom-Presenter will have to handle with large graphic data. If GPU and its memory will handle this data, a great performance enhancement could be reached. Therefore, OpenGL library was used for all image manipulation in Dicom-Presenter. It solves image storing, image manipulations and image rendering. Unfortunately, OpenGL brings 
occasional incompatibility on some hardware which was unacceptable so the main goal of this project is to remove OpenGL from Dicom-Presenter. If OpenGL library would be replaced then also dependencies on Cg toolkit library and plib library would be removed it would highly simplify project compilation and enhance compatibility.



\section{Functionality}
\label{dicom-presenter}
Dicom-Presenter design is based on appearance of other DICOM image viewers. The goal was to display images and perform basic image manipulation. The application window can be divided into two main parts: a rendering part and control part. Rendering part itself is divided into three sub-parts. Most of the place is naturally reserved for image rendering (later called Workspace). Bottom part is used for switching workspaces (later called Workspace Explorer). A side part or rendering window is used for accessing images stored in application memory (later called Image Explorer). Next to the rendering part is a control window which allows user to watch image  properties and adjust them numerically (later called Control Panel). 

Above described concept is very similar to appearance of other viewers unlike the part allowing workspace switching. It is a feature required by IKEM not supported in any of DICOM viewers mentioned in Section \ref{viewers}.

The basic scenario of Dicom-Presenter work session can be described in the following way: After running the application user clicks ``Open Dicom Image'' button located on the control panel. Then the image preview appears in the Image Explorer (see Table \ref{dicompresentergui}). By clicking the image preview, the image becomes a Selected Object. There is only one Selected Object in Dicom-Presenter at the time. A key property of the Selected Object is that the Control Panel contains its information. Therefore, after declaring the opened image as selected a button called ``Create Image Copy'' appears in the Control Panel. By clicking the mentioned button the image appears on the Workspace. If we need to open one more image we click the proper button again. Afterwards a second image preview appears in Image Explorer. Then the necessity of Image Explorer becomes clear.

User is allowed to freely customize the image arrangement on the Workspace. This is a notable feature in Dicom-Presenter. None of the DICOM viewers mentioned in Section \ref{viewers} offers it. Therefore, it was a request from IKEM that image layout has to be fully customizable. Images can be easily moved  by clicking any image at a proper place and dragging it on the Workspace.  It is a user-friendly powerful feature.

As was said before, Dicom-Presenter is aimed to allow presenting images to other physicians. To avoid a need of picture manipulation during a presentation, a possibility of having more workspaces opened was implemented. Each workspace can be prepared before presenting and stored in application memory.
 
\begin{table}[ht]
	\caption{DICOM viewers.}
	\centering
	\begin{tabular}{c}
			\includegraphics[width=0.8\textwidth]{Text/IMG/GUI_Screenshot.png}
		\\
			\includegraphics[width=0.8\textwidth]{Text/IMG/GUI_Screenshot1_English_Label1.png}
		\\
			\includegraphics[width=0.8\textwidth]{Text/IMG/GUI_Screenshot1_English_Label2.png}
		\\
		\end{tabular}
\end{table}%
\chapter*{GUI Applications Programming}
\addcontentsline{toc}{chapter}{GUI Applications Programming}
\vspace{-10mm}

\red{
Co je to GUI?\\
}

\section*{Windows API}
\addcontentsline{toc}{section}{Windows API}

Development of GUI Windows applications is possible with Windows API. It is a name for a set of various APIs - communication interfaces for performing services. These communication interfaces offer manipulation with filesystem, devices and also creating graphic windows.

Most common parts of Windows API are:
\begin{itemize}
\item \clist{kernel32.dll}, which is needed for memory management, input/output, process and thread creation.
\item \clist{user32.dll} for creating GUI elements, such as windows, buttons, etc.
\item \clist{shell32.dll} which offers access to Windows command line
\item \clist{WinSock} to use network
\end{itemize}

Windows API in C++ is used as an ordinary library. It has to be included and linked, then it is possible to call certain functions in program code.

There is an example of simple GUI application on Listing \ref{WinAPI}. \clist{main} function is replaced by \clist{WinMain} function (line \ref{lst:WinMain}). Important part of \clist{WinMain} function is a program loop (line \ref{lst:ProgramLoop}). Windows applications usually run continuously and wait for user input. This is done by infinite program loop where we get system messages including user input. Program behaviour according to user input is here defined in function WndProc (line \ref{lst:WndProc}). There is a switch controller which decides what part of program will be executed.

\begin{lstlisting}[label=WinAPI,caption={An example of a simple application using Windows API for GUI rendering.},escapeinside={@}{@}]
#include <windows.h>
@\label{lst:WndProc}@LRESULT CALLBACK WndProc(HWND hwnd, UINT msg, WPARAM wParam, LPARAM lParam){
    switch(msg){
        case WM_CLOSE:
            DestroyWindow(hwnd);
        break;
        default:
            return DefWindowProc(hwnd, msg, wParam, lParam);
    }
    return 0;
}

@\label{lst:WinMain}@int WINAPI WinMain(HINSTANCE hInstance, HINSTANCE hPrevInstance, LPSTR lpCmdLine, int nCmdShow) {
		WNDCLASSEX wc;
    HWND hwnd;
    MSG Msg;
    wc.lpfnWndProc = WndProc;    
...
    if(not RegisterClassEx(&wc)){
        MessageBox(NULL, TEXT("Window Registration Failed!"), TEXT("Error!"), MB_ICONEXCLAMATION | MB_OK);
        return 0;
    }
    hwnd = CreateWindowEx(
...
		);
    ShowWindow(hwnd, nCmdShow);

@\label{lst:ProgramLoop}@    while(GetMessage(&Msg, NULL, 0, 0) > 0){
        TranslateMessage(&Msg);
        DispatchMessage(&Msg);
    }
    return Msg.wParam;
}
\end{lstlisting}

\section*{GUI Frameworks}
\addcontentsline{toc}{section}{GUI Frameworks}

\red{
Proc bychom meli pouzivat nejaky framework? Vyrazne nam to usnadni praci. Nemusime resit smycku programu. Nemusime resit predavani zprav. Mame predem hotove tridy pro vsechny bezne uzivatelske prvky.\\
Jak nam pouziti frameworku usnadni praci? Ukazat dedictvi puvodnich trid a jejich prizpusobeni nasim pozadavkum.\\
Jak zajistim funkcnost programu, tj. kdyz uzivatel klikne tam a tam stane se to a to? Ukazat system signalu a slotu, ktery je v Qt i v GTK+. Ukazat eleganci a jednoduchost tohoto systemu.\\
}

%\subsection*{Qt Library}
%\addcontentsline{toc}{subsection}{Qt Library}

%Predstavit Qt strucne

\section*{Dicom-Presenter GUI model}
\addcontentsline{toc}{section}{Dicom-Presenter GUI model}

\clist{
Jak je tedy GUI resene v dicom presenteru? Ukazat settings panel, cwidget.\\
Mozna ukazat predavani uzivatelskych zprav z CWidget na workspaceManager, na CWorkspace, na CImage.\\
}
\chapter*{Image representation in computer science}
\addcontentsline{toc}{chapter}{Image representation in computer science}

Dicom-Presenter is an application for viewing image data. Entire image manipulation was solved with use of OpenGL library in early version of Dicom-Presenter. Because OpenGL was expected to be removed from Dicom-Presenter there was a need to reimplement all classes handling image manipulation. Images must be loaded from a file, stored in memory and drew on a screen. According to user need Dicom-Presenter must be able to change image's contrast and brightness. 

\section*{Image representation}
\addcontentsline{toc}{section}{Image representation}




\section*{Image brightness}
\addcontentsline{toc}{section}{Image contrast}

One of important tasks to reimplement in Dicom-Presenter was an ability to change image brightness and contrast. Images which will be opened in Dicom-Presenter can be captured on various MRI units with various imaging properties. The ability to increase image brightness and contrast is mandatory to ensure sufficient display quality. Too dark or too gray images need to be brightened or need to increase contrast to allow obesrvation of smaller physiological findings.



\chapter*{Dicom-Presenter visual engine}
\addcontentsline{toc}{chapter}{Dicom-Presenter visual engine}

\red{
Zde popsat aktualni praci na DP.\\
Zarim jeste nemam rozmyslene poradi sekci.\\
Ale chci popsat zejmena toto:\\
Popsat objektovy model, tedy predstavit tridy: Widget, workspaceManager, Workspace, WorkspaceExplorer, ImageExplorer, Image. Proc mame tyto tridy, jak je to chytre navrzene, ze to kopiruje to, co vidi uzivatel. \\
Vzajemnou spolupraci trid v objektovem modelu, pak lze nejlepe ukazat dvema pohledy:\\
Jak funguje predavani dat pri otevreni snimku.\\
Jak funguje predavani zprav o uzivatelskych akcich mezi tridami.\\
}
\chapter{Multi-planar Reconstruction}
\vspace{-10mm}
\label{multiplanar}



Multi-planar Reconstruction (MPR) in medical imaging is a way of displaying three-dimensional images, captured on an imaging device. A three-dimensional study is displayed in three orthogonal slices. In each slice, there are indicated positions of the other two slices - each slice includes two lines giving the positions (See Figure \ref{fig:multiplanar}). The slices are most often parallel to basic anatomy planes\cite{ctteachingmanual}: sagittal, coronal and transverse plane.  Implementation of Multi-planar Reconstruction was requested by IKEM.

\begin{figure}
 	\caption{Multi-planar reconstruction in Dicom-Presenter.\label{fig:multiplanar}}
	\begin{center}
	\includegraphics[width=0.75\textwidth]{Text/IMG/MultiPlanar.png}
	\end{center}
\end{figure}

\section{Designing an Object Model of Multi-planar Reconstruction}

An implementation of MPR could be divided into three parts:

\begin{itemize}
\item Handling user control. %User has to be able to freely manipulate with the three planes. Is existing
\item Image Rendering. %Is it possible to use existing classes to render three planes of a DICOM study?
\item Integration to existing object model. %The questions are, where to place the new class in the existing class hierarchy (see Section \ref{dpobjectmodel}) and how much functionality of the new class can be inherited from existing classes.
\end{itemize}

The question to discuss in all three steps is, how much of existing functionality can be used and what parts are necessary to be implemented. It is possible to implement a completely new class, which will handle user control and rendering itself. But it would lead to multiple implementations of similar tasks. The aim is to use maximum of existing functionality.

The functionality of new classes performing Multi-planar reconstruction will be similar to existing classes: Workspace and Image. Workspace class manages placement of images on the computer screen - similarly, three slices of MPR will be placed on computer screen. There is a process of obtaining proper slice from a three-dimensional image in the Image class. When using MPR, three slices will be taken in three orthogonal planes.

The similarities to Image and Workspace classes could be solved in several ways:

\begin{itemize}
\item It is possible to create a new standalone class, which will have partially similar functionality to the existing one.
\item The new functionality can be added to the existing class.
\item The existing class can be inherited by the new class.
\item An identical functionality of the class and existing class can be extracted into an abstract class, which will be inherited by both of them.
\end{itemize}

The most importance was given to the fourth option, because unlike the first option it does not add source code repetition. In addition it does not lead to creating classes with great volume of rarely used source code unlike the second option.

\section{Implementation Process of Multi-planar Reconstruction}

The class of Multi-planar Reconstruction (further MPR Workspace class), which will hold simillar function as the Workspace, was implemented as a new standalone class. The class has firmly defined layout of including images and respones to mouse events are re-defined. The MPR class owns one object of Image type. New functions for MPR rendering were added to the existing Image class.

This solution is fully functional, but it has two disadvanteges:

\begin{itemize}
\item Firsly, the MPR functions added to the Image class were constantly unused outside MPR.
\item Secondly, a relation between a Workspace and Images is different to relation between MPR Workspace and displayed Images. The MPR Workspace includes one Image object, which is responsible for rendering all three slices. More reasonable solution would be, that MPR Workspace would own three Image objects, each of them responsible for rendering one slice. This scheme corresponds to previous understandings of Image and Workspace. But moreover it would allow the three slices to by fully adjustable like an ordinary Image object (position, size, zoom, ...). 
\end{itemize}

To remove the first complaint, it will be needed to add the MPR functions to a new class and inherit the existing Image class. To avoid the second disadvantage, it will be needed to change the object design of MPR:

\begin{itemize}
\item The responsibilities of processing the user input will be shifted from the MPR Workspace class to the Image class.
%\item Then, it will be possible to allow moving and resizing the MPR Images along the Workspace easily\footnote{The Image position will be bound to the Image GUI.}.
\item The next step will be creating an abstract class describing the positions of all three slices.
\item Afterwards, three Image classes will be used for rendering the MPR. Then, the images will be fully accessible for manipulation.
\end{itemize}




\chapter{Project Build Environment}
\vspace{-10mm}
\label{compilationchapt}
Compilation of larger projects such as Dicom-Presenter should be divided into parts and the linking process should be described in some form. Moreover, external libraries need to be compiled in a compatible configuration. This Chapter sets a goal to describe compilation process of larger projects.

\section{Application Modules}

\label{library}

A C++ application is not usually compiled entirely in one step. It is divided into smaller logic parts known as modules. These parts are compiled individually and then they are linked into the executable at the end. The benefit is that, if the source code of the application is changed, it is not needed to recompile the whole application - only the modules including the affected source code are recompiled. 


\section{CMake}

CMake\cite{cmake_home} allows description of the compilation process. It specifies mainly:

\begin{itemize}
\item Grouping project files into project libraries.
\item Dependency on external libraries.
\item Description of custom build steps.
\end{itemize}

The most common tools for automated compilation are GNU Automake on Linux OS and Microsoft Visual Studio projects on Windows OS\cite{msvcompilation}. CMake is more general tool, it allows for the generation of both: GNU Automake files and Microsoft Visual Studio project. Since Dicom-Presenter was supposed to be a multi-platform application, CMake was used for automating the compilation process.

CMake is often used in open-source projects. Both plib\citesec{plibhome} and dcmtk\citesec{dcmtkhome} libraries, used in Dicom-Presenter, were equipped with a CMake for script.

\subsection{CMake Syntax}


The following section describes CMake syntax necessary to build simple applications. 

A CMake build description is always written in a file named \clist{CMakeLists.txt} and placed in the project root directory. At the beginning of the file there has to be preamble requesting CMake of newer version than declared: \clist{cmake\-\_minimum\-\_required\-(VERSION ...)}.

When compiling from a command line with use of gcc or g++ it is necessary to mention where the compiler can find header files from external libraries and where the linker can find .lib files from external libraries. CMake uses commands \texttt{include\-\_directories\-("...")} and \texttt{link\-\_directories\-("...")} for defining paths to libraries.

As mentioned in Section \ref{library}, C++ applications are firstly compiled in parts into internal libraries and then linked together. A library compilation is in CMake declared by command \texttt{add\-\_library\-(library1 library1.cpp library1.h ...)}. The first parameter is the name of the library. It is the name of the output file, but moreover it is also the name under which CMake remembers this library. The other parameters make a list of files which will be compiled as the library.

If a library depends on external libraries a command \texttt{target\-\_link\-\_libraries\-(library1 lib1 lib2)} defines the dependencies. The first parameter is the name of the internal library. The other parameters are filenames of external libraries.

It is possible to add custom build commands with statement: {\tt add\-\_custom\-\_command\-(COMMAND command1 OUTPUT outputfile1)}. A shell command follows a \clist{COMMAND} macro. It can be any Linux or Windows console command including a program call. The second argument placed after \clist{OUTPUT} macro is a label for an output file generated by the command. Then, this file can be used later on in the CMake script. It can be a source-code file, resource file, library, etc.




\definecolor{LightGray}{RGB}{245,245,245}
\definecolor{LightRed}{RGB}{150,150,150}
\definecolor{LightGreen}{RGB}{0,0,0}
\definecolor{LightBlue}{RGB}{100,100,100}
\lstset{ %
language=C++,                % choose the language of the code
basicstyle=\tt\small\color{LightBlue},          % print whole listing small
keywordstyle=\small\color{black},	% bold black keywords
identifierstyle=\small\color{LightBlue},           % nothing happens
commentstyle=\small\color{Rhodamine}, % white comments
stringstyle=\ttfamily,      % typewriter type for strings
showstringspaces=false,     % no special string spaces
numbers=left,                   % where to put the line-numbers
numberstyle=\tiny\tt,      % the size of the fonts that are used for the line-numbers
%stepnumber=2,                   % the step between two line-numbers. If it's 1 each line will be numbered
numbersep=5pt,                  % how far the line-numbers are from the code
%backgroundcolor=\color{LightGray},  % choose the background color. You must add \usepackage{color}
showspaces=false,               % show spaces adding particular underscores
showstringspaces=false,         % underline spaces within strings
showtabs=false,                 % show tabs within strings adding particular underscores
frame=single,			% adds a frame around the code
tabsize=3,	                % sets default tabsize to 2 spaces
%captionpos=b,                   % sets the caption-position to bottom
breaklines=true,                % sets automatic line breaking
breakatwhitespace=false,        % sets if automatic breaks should only happen at whitespace
%title={Zdrojovy kod},                 % show the filename of files included with \lstinputlisting; also try caption instead of title
%escapeinside={\%*}{*)}          % if you want to add a comment within your code
escapechar=!,
}

\begin{lstlisting}[caption={CMake script for application compilation.}, language=make, morekeywords={cmake_minimum_required, project, set, include_directories, link_directories, add_custom_command, OUTPUT, COMMAND, add_library, target_link_libraries, ADD_EXECUTABLE},keywordstyle=\small\color{LightGreen}]
cmake_minimum_required(VERSION 2.8)
project (dicom-presenter)

set (Qt_LIBRARY_PATH /usr/share/qt4)
set (Cg_LIBRARY_PATH /usr)
...

include_directories (!\redlist{``./src''}!)
include_directories (!\redlist{``\${Qt\_LIBRARY\_PATH}/include''}!)
include_directories (!\redlist{``\${Cg\_LIBRARY\_PATH}/include''}!)
...

link_directories (!\redlist{``\${Qt\_LIBRARY\_PATH}/lib''}!)
link_directories (!\redlist{``\${Cg\_LIBRARY\_PATH}/lib''}!)
...

add_custom_command (OUTPUT moc/moc_mainWindow.cpp COMMAND moc src/mainWindow.h > moc/moc_mainWindow.cpp)
add_library (mainWindow src/mainWindow.cpp moc/moc_mainWindow.cpp src/mainWindow.h)
target_link_libraries (mainWindow QtGui GLEW)

add_library (glImage src/glObjects/glImage.cpp src/glObjects/glImage.h)
target_link_libraries (glImage QtOpenGL QtCore)

...

ADD_EXECUTABLE (dp src/main.cpp)
target_link_libraries (dicom-presenter mainWindow glImage ...)
\end{lstlisting}

\section{Libraries}
Some parts of application's source code, which could be used in other programs, can be compiled into reusable units - libraries. C++ distinguishes two concepts of libraries: static-link and dynamic-link libraries:

\begin{itemize}
\item Static linking of a library means that all functions of the library are copied into the executable. Firstly, a library's source code is compiled into a .lib file and then it is inserted as a whole into the executable file.
\item Dynamic linking of a library means that only the library's name and its function names are inserted into the executable. An executable code of the linked library is distributed together with the executable file in a standalone .dll file. When the application is run the operating system has to load the library's executable code into the application's virtual memory. While compiling a dynamic-link library the compiler produces also a .lib file. The executable is linked against this .lib file but the lib file contains only instructions to load machine code from .dll files.
\end{itemize}

The concept of dynamic-link libraries shows its potential when applied to libraries that are used by multiple applications. Dynamic-link libraries are also called shared libraries. These are the advantages of using shared libraries:

\begin{itemize}
\item If a dynamic-link library is used by multiple applications it is enough to be loaded once in computer memory. Multiple applications can access library functions at the same time.
\item If a library needs to be updated, there's no need to recompile the applications using the library as long as the library's function declarations are not changed.
\end{itemize}


\section{Run-time Loaded Libraries}

Besides linking against a dynamic-link library, C++ environment offers a run-time loading of a shared library. A module containing the required function is loaded into the computer memory by calling WinAPI functions. Run-time loading is useful for modules that are used infrequently. Unlike compilation-time linking, the run-time loaded modules do not need to be stored in the computer memory all the application run-time. The modules can be loaded only if used and then removed from the computer memory.

A disadvantage of run-time library loading is the amount of necessary declarations and function calls to be undertaken, before a module is loaded.

\subsection{Compilation of Run-time Loaded Libraries}


A run-time loaded \clist{.dll} library is very similar to an \clist{.exe} binary, but there is a function exports table in a \clist{.dll} library\cite{msdn}. Exported functions are accessible from outside of the library, functions not exported are private functions. Therefore, a compiler needs to know which functions have to be exported. There are two ways for declaring this: a \clist{dllexport} macro can be used or a \clist{Module Definition File} can be used. A \clist{Module Definition File} is a file including a list of exported functions using a very simple syntax. There is an example of a \clist{Module Definition File} in Listing \ref{DEF}. An example of using the \clist{dllexport} macro can be seen in Listing \ref{dllspec}. Placing the macro before a function definition is enough to tell a compiler to export the function.

\begin{lstlisting}[label=DEF,caption={A \clist{Module Definition File} of a library called ``Mathfuncs'' including a function called ``PrimeTest''.}]
LIBRARY   MATHFUNCS
EXPORTS
   PrimeTest	@1
\end{lstlisting}


\begin{lstlisting}[label=dllspec,caption={An example of exporting a function by using a \clist{dllexport} macro.},escapeinside={@}{@}]
#include <iostream>
#include <cmath>
#include "mathfuncs.h"

extern "C"{
	__declspec(dllexport) bool PrimeTest(int n){
	bool isPrime=true;
	for (int d=2;d<=sqrt((float)n);d++){
		if (n%d==0) isPrime=false;
	}
	return isPrime;
}
\end{lstlisting}

The function \clist{PrimeTest} (Listing \ref{dllspec}) will be accessible from outside of the library.

\subsection{Run-time Library Loading}

\label{sec:runtimeloading}

Functions from run-time loaded libraries are accessed through function pointers\cite{msdn}. Therefore, a function pointer of a given type must be declared at the time of compilation. The type of the function pointer denotes which input parameters are passed to the function and which variable type will be returned by the function. C/C++ language does not offer type definitions at the runtime, therefore the type of the function pointer must be defined at the compilation time. The Listing \ref{runtime} includes a function type declaration at line \ref{lst:typedef}.

Before the desired function from the external library can be accessed, the library module must be loaded into the RAM memory. The library is loaded by calling a \clist{LoadLibrary} function from WinAPI, besides, Qt library offers \clist{QLibrary} class which encapsulates platform-dependent function calls.

After loading the library into the computer memory, a specific memory address of the accessed function must be obtained. Owing to the compilation time function export, the WinAPI or Qt library is able to acquire the exact address of the function. A \clist{GetProcAddress} function from WinAPI or \clist{QLibrary::resolve} function from Qt library can be used (line \ref{lst:getproc} on Listing \ref{runtime}).

Then, after assigning the obtained address to the function pointer, the accessed function can be called at any time until unloading the library from the computer memory (lines \ref{lst:pointer1}, \ref{lst:pointer2} and \ref{lst:pointer3} on Listing \ref{runtime}).

\begin{lstlisting}[label=runtime,caption={An application loading a .dll library at runtime.},escapeinside={@}{@}]
#include <iostream>
#include <windows.h>

@\label{lst:typedef}@typedef bool PrimeTestFunction(int);

int main (){
	HINSTANCE LoadedLibrary;
	LoadedLibrary=LoadLibrary(TEXT("MathFuncs.dll"));
	if (LoadedLibrary!=NULL){
		FARPROC ProcessAdress;
		@\label{lst:getproc}@ProcessAdress=GetProcAddress(LoadedLibrary,"PrimeTest");
		@\label{lst:pointer1}@PrimeTestFunction* _PrimeTestFunction=(PrimeTestFunction*)ProcessAdress;
		if(_PrimeTestFunction){
		@\label{lst:pointer2}@	std::cout << "23 is prime number:" << _PrimeTestFunction(23) << std::endl;
		}
		@\label{lst:pointer3}@FreeLibrary(LoadedLibrary);
	}
}
\end{lstlisting}



\section{C++ Standard Library in MS Visual Studio}

\label{standardlibrary}
C++ Standard Library\footnote{C++ Standard library is a C++ version of C Run-time library. Unfortunately both names are often confused.} is used in almost all C++ application. It provides communication with the operation system. It offers manipulation with standard input and output (stdio.h), memory allocation (stdlib.h), math functions (math.h), time related functions (time.h), etc. If the library is used in application its header files are included in the source code and the library is linked into an executable file.

There are four different implementations of C++ Standard library in Microsoft Visual Studio compiler. The rule is that only one type of implementation can be linked into an executable file. The problem is if multiple external libraries are used in one project. Every external library is compiled with the use of some kind of Standard C++ library implementation. If we try to link an application against external libraries using different implementations of C++ Standard library we receive a linker error 2005 - already defined. This error arises because multiple versions of C++ Standard library are linked at the same time but they define the same functions.

Therefore, it is important to distinguish between implementations of C++ Standard library. All external libraries as well as all inner libraries have to be linked against the same version of C++ Standard library! Standard C++ library implementations differ according to use of dynamic or static linking and according to debug or release building. So, all used libraries have to use static or dynamic linking and all of them have to be built in debug or release mode. Then, the final application is able to be linked. Therefore, if we use external libraries we often have to rebuild them to gain the appropriate version of the library using required version of C++ Runtime library.

In addition to the problem explanation the Table \ref{standardlibrarytable} is included. It should help to resolve the purpose of problem when he linker error 2005 is received. 

\begin{table}
    \caption{A list of four build configurations using different implementation of C++ Standard Library.\label{standardlibrarytable}}
\begin{center}
	\begin{tabular}{| l| l | l |}
	  \hline                       
	  Link type & Build mode & Library file \\
	  \hline
	  \hline                     
	  Static & Debug & LIBCPMTD.LIB\\
	  \hline
	  Static & Release & LIBCPMTD.LIB\\
	  \hline  
	  Dynamic & Debug & MSVCPRTD.LIB\\
	  \hline  
	  Dynamic & Release & MSVCPRT.LIB\\  
	  \hline  
	\end{tabular}
\end{center}
\end{table}





\chapter{Plugin system}
\vspace{-10mm}

There was a need of image segmentation support in Dicom-Presenter. A few of FNSPE student are working on image segmentation algorhitms of MRI images. Therefore raised an idea to import these algorhitms into Dicom-Presenter. The best idea was to design a plugin system for importing segmentation algorhitms.

Plugin in computer sciences is an optional addition to some application. It is not usually distributed with application itself, but can be added by user according to his needs. An example of a plugin can be an additional python script to Gimp program which allows user to apply 'sepia effect' on his photos. According to the mentioned example, plugins can be written in another language than application itself. They obtain some new functionality to application.

All segmentation algorithms developed at FNSPE were written in C language due to its performance. Plugins in C language are compiled to a Dynamic-Link libraries and run-time loaded. A theory of C++ libraries and description of run-time loading libraries can be found in Section \ref{library}.

\section{Image segmentation algorithms}

\red{
Popsat algoritmy Kuby Louckeho a Radka Maci.\\
}

\section{Plugins system requirements}

It is needed to set rules for plugin libraries. Therafter, a library following these rules should be applicable in DicomPresenter. The first thing is to generalize what the libraries will need as an input and output. All three libraries made by FNSPE students required an image with designated segmentable area together with computational parameters as an input. The most reasonable way to insert computation parameters is to generate a GUI interface including required input elements. This chapter discusses plugins GUI construction done at the application runtime and its connection to the loaded plugin library. A plugin interface was considered to be as simple as possible - a key idea was to require a minimum of source code modifications in existing image segmentation libraries.

\section{Generating plugins GUI}

Dicom-Presenter's graphic user interface is builded with use of Qt framework. Objects of Qt library classes represent application control elements. The objects can be created and displayed at a runtime according to plugin needs. A mandatory question is which way to choose for plugin GUI description.

Qt library offers its own format of GUI description. There is a tool in Qt SDK\footnote{Qt Software Development Kit is a package for Qt based applications development. SDK includes library files together with a native development IDE.} which offers an interactive GUI creation. It produces a XML file which describes the GUI. A strong argument for using this solution is that Qt library offers native tools for processing generated GUI description files.

Another possible way is to declare own language for GUI description. Firstly, it was implemented easily: Lines in a description file were corresponding to GUI elements. First word of a line determined a type of an element. Other words on the line described element behaviour such as default value, minimum and maximum value or matching function in a library. The second step was a transformation to XML standards. The result was simillar to preceding option but distinctive in sytax and naming. 

It would seem that the first solution is more reasonable. No new language is declared, though existing Qt library standard is used. Unfortunately, the Qt library syntax was primarily supposed to be generated by automated tool. The XML code is too complicated to be written manually - there is too much required phrases in the code. Morover, GUI elements are described by class-names from Qt library - therefore, it is needed to know the Qt library to be able to write a description manually. The only option would be to generate GUI description by the Qt interactive tool. However, it is a complex tool, aimed to creating extensive GUI layouts for large applications. Unlike, segmentation algorithm plugins require only a few control elements (input fields and buttons). It seems more accessible for a programmer to follow a few rules declared in GUI description language created for Dicom-Presenter instead of following extensive number of rules attached to Qt GUI description language.

\subsection{Dicom-Presenter GUI description language}

The GUI description should determine types of control elements their layout in interface window. The language for Dicom-Presenter GUI description is supposed to be simple. For every numeric input field it is needed to know: a required variable type (integer or float), the minimal and maximal possible value, the default value, a parameter description. Passing char parameters was realised by a combobox element - all possible values are listed in a dropdown menu.

A numeric field is declared by a string:

\clist{<numinput type="..." min="..." max="..." default="..." name="..."/>}

Where \clist{type} determines a C/C++ type (integer of double), \clist{min}, \clist{max}, \clist{default} determine minimum, maximum and default value and \clist{name} determines a parameter description.

A combobox can be declared in a similar way like in (X)HTML:

\noindent \indent \clist{<combobox type="..." name="...">}\\
\indent \indent \clist{<option value="..." name="..."/>}\\
\indent \indent \clist{<option value="..." name="..."/>}\\
\indent \clist{</combobox>}

Where \clist{type} determines a variable type passed to a library. \clist{name} in \clist{combobox} tag is a text description of property. The \clist{name} property in \clist{option} tag is a text description of a \clist{value} which is passed to a library.

\section{Dynamically loading plugin functions}

Dicom-Presenter's image segmentation algorithms use individual parameters. Therefore, each main computational function in each plugin needs individual number and types of parameters. A C/C++ function located in dynamically loaded library is ran through a function pointer as seen on Listing \ref{sec:runtimeloading}. This pointer must have a fixed number and fixed types of parameters given at compiliation. Thus, it is not possible to call various functions with various sets of parameters from a dynamically loaded library. It is possible to run only a finite number of types of functions according to a set of given parameters. There were several solutions how to make the Dicom-Presenter plugin system variable enough to load and run previously unknown functions:

\begin{enumerate}
\item It is possible to pre-define sufficient number of function pointer types at a compilation time. Then, each Dicom-Presenter plugin must have input function of predefined type.
\item Another solution is based on a fact that a function pointer can have more parameters than targeted function. For example, a function pointer having five integer parameters can point to a function receiving only three integer parameters - the last two given parameters will be ignored. Therefore it would be possible to define function pointers with excessing number of float, integer and char parameters. A function pointer would be assigned to a function to fit the first couple of parameters and ignoring the rest.
\item Another applicable way is to allow only one-parameter requiring functions to be present in a plugin. If the algorithm would need more parameters, then appropriate functions settings these parameters would be ran. Parameters can be saved as global variables to be reachable by main computational function. This solution has a great advantage: It is possible to call the parameter setting functions simultaneously to user  manipulating with relevant GUI elements. Then, an instant feedback from the library could be present (checking for incompatible values). This solution was implemented but it seemed too restrictive for library design. 
\item Last option is inspired by passing command line arguments to a standard C/C++ application. C/C++ Main Function receives a pointer to an array of all arguments. As in this case a plugin library function can receive three pointers to three array types: pointer to int, pointer to double and pointer to char. The arrays can be of any length.
\end{enumerate}

At first the third option was used in implementation. Function setting algorithm parameters were called while setting parameters in GUI. But there were no real benefits of library response while setting GUI parameters. More than that the rules for plugins were too bounding. Therefore, the final version of plugins API were implemented using option 4.

\section{User Interaction in Image Segmentation plugins}

\red{
Popsat jak uzivatel musi nakreslit seminku krivky.\\
Popsat implementaci, jaka byla pouzita.\\
}
%%%%%%%%%%%%%%%%%%%%%%  LITERATURA  %%%%%%%%%%%%%%%%%%%%%%

\nocite{*}				% zobrazuj i necitovane primarni zdroje
\bibliography{Text/Literature}
\nocitesec{*}				% sek. zdroje
\bibliographysec{Text/Sec}


\end{document}

